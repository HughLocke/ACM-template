\documentclass[a4paper,11pt]{article}
\usepackage{zh_CN-Adobefonts_external}
\usepackage{fancyhdr}  % 页眉页脚
\usepackage{minted}    % 代码高亮
\usepackage[colorlinks]{hyperref}  % 目录可跳转
\usepackage{geometry}
\geometry{left=2cm,right=1cm,top = 2cm,bottom = 2cm}
\setlength{\headheight}{15pt}
\setlength{\parindent}{1em}
% 定义页眉页脚
\pagestyle{fancy}
\fancyhf{}
\fancyhead[C]{ACM Template of Hugh Locke}
\lfoot{}
\cfoot{\thepage}
\rfoot{}

\author{Hugh Locke}   
\title{ACM模板}

\begin{document} 
\maketitle % 封面
\newpage % 换页

\tableofcontents % 目录
\newpage
\section{一切的开始} 
\subsection{头文件与预处理} 
\inputminted[breaklines]{c++}{其他/start.cpp}
\subsection{cin关同步}
\inputminted[breaklines]{c++}{其他/cin关同步.cpp}
\subsection{读入挂}
\inputminted[breaklines]{c++}{其他/读入挂.cpp}

\section{图论}

\subsection{图论的基础}
\subsubsection{链式前向星}
\inputminted[breaklines]{c++}{图论/init.cpp}
%通用
\subsection{最短路}
\subsubsection{堆优化Dijkstra}
\inputminted[breaklines]{c++}{图论/Dijkstra.cpp}
\subsubsection{SPFA}
\inputminted[breaklines]{c++}{图论/SPFA.cpp}
\subsubsection{Floyd}
\inputminted[breaklines]{c++}{图论/Floyd.cpp}
\subsection{分层图}
\inputminted[breaklines]{c++}{图论/分层图.cpp}
\subsection{差分约束}
\inputminted[breaklines]{c++}{图论/差分约束.cpp}
\subsection{欧拉路径}
\inputminted[breaklines]{c++}{图论/欧拉路径.cpp}
\subsection{kruskal重构树}
\inputminted[breaklines]{c++}{图论/kruskal重构树.cpp}
\subsection{Tarjan}
\subsubsection{割点}
\inputminted[breaklines]{c++}{图论/割点.cpp}
\subsubsection{缩点}
\inputminted[breaklines]{c++}{图论/缩点.cpp}
\subsubsection{点双}
\inputminted[breaklines]{c++}{图论/点双.cpp}
\subsection{2-SAT}
\subsubsection{缩点法}
\inputminted[breaklines]{c++}{图论/2-SAT缩点法.cpp}
\subsubsection{染色法}
\inputminted[breaklines]{c++}{图论/2-SAT染色法.cpp}
\subsection{斯坦纳树}
\inputminted[breaklines]{c++}{图论/斯坦纳树.cpp}
%网络流
\subsection{二分图匹配}
\inputminted[breaklines]{c++}{图论/二分图匹配.cpp}
\subsection{KM算法}
\inputminted[breaklines]{c++}{图论/KM算法.cpp}
\subsection{最大流}
\subsubsection{Dinic}
\inputminted[breaklines]{c++}{图论/Dinic.cpp}
\subsubsection{最小割}
\inputminted[breaklines]{c++}{图论/最小割.cpp}
\subsubsection{最大权闭合子图}
\inputminted[breaklines]{c++}{图论/最大权闭合子图.cpp}
\subsection{最小覆盖}
\subsubsection{DAG最小点路径覆盖}
\inputminted[breaklines]{c++}{图论/最小点路径覆盖.cpp}
\subsection{费用流}
\subsubsection{SPFA费用流}
\inputminted[breaklines]{c++}{图论/SPFA费用流.cpp} 

\subsection{有上下界的网络流}
\subsubsection{无源汇上下界可行流}
\inputminted[breaklines]{c++}{图论/无源汇上下界可行流.cpp} 
\subsubsection{有源汇上下界网络流}
\inputminted[breaklines]{c++}{图论/有源汇上下界网络流.cpp} 
%树上相关
\subsection{树的重心}
\inputminted[breaklines]{c++}{图论/树的重心.cpp}
\subsection{树的直径}
\subsubsection{双dfs法}
\inputminted[breaklines]{c++}{图论/树的直径双dfs.cpp}
\subsubsection{树形dp法}
\inputminted[breaklines]{c++}{图论/树的直径树dp.cpp}
\subsection{树上k半径覆盖}
\inputminted[breaklines]{c++}{图论/树上k半径覆盖.cpp}
\subsection{dfs序}
\inputminted[breaklines]{c++}{图论/括号序列.cpp}
\subsection{括号序列}
\inputminted[breaklines]{c++}{图论/括号序列.cpp}
\subsection{LCA}
\inputminted[breaklines]{c++}{图论/LCA.cpp} 
\subsection{点分治}
\inputminted[breaklines]{c++}{图论/点分治.cpp}
\subsection{动态点分治}
\inputminted[breaklines]{c++}{图论/动态点分治.cpp}
\subsection{树上差分}
\inputminted[breaklines]{c++}{图论/树上差分.cpp}
\subsection{树链剖分} 
\inputminted[breaklines]{c++}{图论/树链剖分.cpp}
%小贴士
\subsection{图论小贴士} 
\input{图论/图论小贴士.txt}

\section{数据结构} 

\subsection{链表}
\inputminted[breaklines]{c++}{数据结构/链表.cpp} 

\subsection{并查集} 
\subsubsection{可撤销并查集} 
\inputminted[breaklines]{c++}{数据结构/可撤销并查集.cpp}

\subsubsection{可持久化并查集} 
\inputminted[breaklines]{c++}{数据结构/可持久化并查集.cpp} 

\subsubsection{种类并查集}
\inputminted[breaklines]{c++}{数据结构/种类并查集.cpp} 

\subsection{启发式合并}
\inputminted[breaklines]{c++}{数据结构/启发式合并.cpp} 

\subsection{ST表}
\inputminted[breaklines]{c++}{数据结构/ST表.cpp}

\subsection{树状数组}
\inputminted[breaklines]{c++}{数据结构/树状数组.cpp}

\subsection{二维平面} 
\subsubsection{二维前缀和} 
\inputminted[breaklines]{c++}{数据结构/二维前缀和.cpp}
\subsubsection{二维ST表} 
\inputminted[breaklines]{c++}{数据结构/二维ST表.cpp}

\subsection{莫队算法} 
\subsubsection{无修莫队} 
\inputminted[breaklines]{c++}{数据结构/无修莫队.cpp}

\subsubsection{回滚莫队} 
\inputminted[breaklines]{c++}{数据结构/回滚莫队.cpp}

\subsubsection{带修莫队} 
\inputminted[breaklines]{c++}{数据结构/带修莫队.cpp}

\subsubsection{树上莫队} 
\inputminted[breaklines]{c++}{数据结构/树上莫队.cpp}

\subsection{珂朵莉树}
\inputminted[breaklines]{c++}{数据结构/珂朵莉树.cpp}

\subsection{动态开点线段树}
\inputminted[breaklines]{c++}{数据结构/动态开点线段树.cpp}

\subsection{李超树}
\inputminted[breaklines]{c++}{数据结构/李超树.cpp}

\subsection{左偏树}
\inputminted[breaklines]{c++}{数据结构/左偏树.cpp}

\subsection{Splay}
\inputminted[breaklines]{c++}{数据结构/Splay.cpp}

\subsection{LCT}
\inputminted[breaklines]{c++}{数据结构/LCT.cpp}

\subsection{主席树}
\subsubsection{静态区间第k小}
\inputminted[breaklines]{c++}{数据结构/静态区间第k小.cpp}

\subsection{cdq分治}
\subsubsection{三维偏序问题}
\inputminted[breaklines]{c++}{数据结构/三维偏序.cpp}

\subsection{kd树}
\inputminted[breaklines]{c++}{数据结构/kd树.cpp}

\section{字符串}

\subsection{Hash}
\subsubsection{字符串Hash}
\inputminted[breaklines]{c++}{字符串/字符串Hash.cpp} 

\subsubsection{图上Hash}
\inputminted[breaklines]{c++}{字符串/图上Hash.cpp} 

\subsection{KMP} 
\subsubsection{KMP} 
\inputminted[breaklines]{c++}{字符串/KMP.cpp} 
\subsubsection{EXKMP}
\inputminted[breaklines]{c++}{字符串/EXKMP.cpp} 

\subsection{Shift-And}
\inputminted[breaklines]{c++}{字符串/Shift-And.cpp} 


\subsection{Manacher}
\inputminted[breaklines]{c++}{字符串/Manacher.cpp} 

\subsection{回文树}
\inputminted[breaklines]{c++}{字符串/回文树.cpp} 

\subsection{Trie树}
\inputminted[breaklines]{c++}{字符串/Trie树.cpp} 

\subsection{AC自动机}
\inputminted[breaklines]{c++}{字符串/AC自动机.cpp} 

\subsection{序列自动机}
\inputminted[breaklines]{c++}{字符串/序列自动机.cpp} 

\subsection{fail树} 
\inputminted[breaklines]{c++}{字符串/fail树.cpp} 

\subsection{后缀数组} 
\inputminted[breaklines]{c++}{字符串/后缀数组.cpp} 

\subsection{后缀自动机} 
\inputminted[breaklines]{c++}{字符串/后缀自动机.cpp} 

\section{动态规划}
\subsection{悬线法dp} 
\inputminted[breaklines]{c++}{动态规划/悬线法dp.cpp} 

\subsection{斜率优化dp}
\inputminted[breaklines]{c++}{动态规划/斜率优化dp.cpp} 

\subsection{数位dp}
\inputminted[breaklines]{c++}{动态规划/数位dp.cpp} 

\subsection{错排公式}
\inputminted[breaklines]{c++}{动态规划/错排公式.cpp} 

\section{数论} 

\subsection{gcd} 
\inputminted[breaklines]{c++}{数论/gcd.cpp}

\subsection{exgcd} 
\inputminted[breaklines]{c++}{数论/exgcd.cpp}

\subsection{逆元} 
\inputminted[breaklines]{c++}{数论/逆元.cpp}

\subsection{卡特兰数} 
递归公式1
\begin{displaymath}
f(n)=\sum_{i=0}^{n-1}f(i)*f(n-i-1)
\end{displaymath}

递归公式2
\begin{displaymath}
f(n)=\frac{f(n-1)*(4*n-2)}{n+1}
\end{displaymath}

组合公式1
\begin{displaymath}
f(n)=\frac{C_{2n}^n}{n+1}
\end{displaymath}

组合公式2,重要!重要!重要!
\begin{displaymath}
f(n)=C_{2n}^n-C_{2*n}^{n-1}
\end{displaymath}

递推公式
\begin{displaymath}
f[n]=\sum_{i=0}^{n-1}f[i]*f[n-i-1]
\end{displaymath}
\inputminted[breaklines]{c++}{数论/卡特兰数.cpp}

\subsection{Miller-robin} 
\inputminted[breaklines]{c++}{数论/Miller-robin.cpp} 

\subsection{0/1分数规划} 

如一个物品有成本bi和价值ai的时候,我们需要一种方案使得取K的物品之后价值和成本的比例最高
\begin{displaymath}
R=\sum \frac{{a}_{i}*{x}_{i}}{{b}_{i}*{x}_{i}}
\end{displaymath}
形如式子中选定每个xi的值为0/1,最终使得R最大\\
解:二分出答案t按来代入式子取check,每次选取ai - bi * t最大的K个物品,如果最终∑ai - bi * t >= 0,说明答案t可行

\subsection{SG函数} 
\inputminted[breaklines]{c++}{数论/SG函数.cpp} 

\subsection{中国剩余定理} 
\inputminted[breaklines]{c++}{数论/中国剩余定理.cpp}

\subsection{bsgs} 
\inputminted[breaklines]{c++}{数论/bsgs.cpp}

\subsection{组合数} 
\inputminted[breaklines]{c++}{数论/组合数.cpp}

\subsection{卢卡斯定理} 
\inputminted[breaklines]{c++}{数论/卢卡斯定理.cpp}

\subsection{康拓展开} 
\inputminted[breaklines]{c++}{数论/康拓展开.cpp} 

\subsection{母函数} 
\inputminted[breaklines]{c++}{数论/母函数.cpp} 

\subsection{高斯消元} 
\inputminted[breaklines]{c++}{数论/高斯消元.cpp} 

\subsection{线性空间} 
\inputminted[breaklines]{c++}{数论/线性空间.cpp}

\subsection{线性基} 
\inputminted[breaklines]{c++}{数论/线性基.cpp}

\subsection{拉格朗日插值}
\subsubsection{拉格朗日插值}
\begin{displaymath}
f(k) = \sum_{i = 0}^{n}y_i\prod_{i != j}^{}\frac{k - x[j]}{x[i] - x[j]}
\end{displaymath}
\inputminted[breaklines]{c++}{数论/拉格朗日插值.cpp}

\subsubsection{在x取值连续时的做值}
\begin{displaymath}
pre_i = \prod_{j = 0}^{i} k - j
\end{displaymath}
\begin{displaymath}
suf_i = \prod_{j = i}^n k - j
\end{displaymath}
\begin{displaymath}
f(k) = \sum_{i=0}^n y_i \frac{pre_{i-1} * suf_{i+1}}{fac[i] * fac[n - i]}
\end{displaymath}

\subsubsection{重心拉格朗日插值法}
\begin{displaymath}
g = \prod_{i=1}^n k - x[i]
\end{displaymath}
\begin{displaymath}
t_i = \frac{y_i}{\prod_{j \not =i} x_i - x_j}
\end{displaymath}
\begin{displaymath}
f(k) = g\sum_{i = 0}^{n}  \frac{t_i}{(k - x[i])}
\end{displaymath}
\inputminted[breaklines]{c++}{数论/重心拉格朗日插值法.cpp}
\subsubsection{自然数连续幂次和}
\inputminted[breaklines]{c++}{数论/自然数连续幂次和.cpp}

\subsection{FFT} 
\subsubsection{卷积} 
\inputminted[breaklines]{c++}{数论/fft卷积.cpp} 
\subsubsection{递归法} 
\inputminted[breaklines]{c++}{数论/fft递归法.cpp} 
\subsubsection{迭代法} 
\inputminted[breaklines]{c++}{数论/fft迭代法.cpp} 
\subsubsection{字符串匹配} 
\inputminted[breaklines]{c++}{数论/fft字符串匹配.cpp} 
\subsubsection{NTT} 
\inputminted[breaklines]{c++}{数论/NTT.cpp} 
\subsection{FWT}
\inputminted[breaklines]{c++}{数论/FWT.cpp}

\subsection{反演定理} 
\subsubsection{二项式反演} 
\inputminted[breaklines]{c++}{数论/二项式反演.cpp}
\subsubsection{莫比乌斯反演} 
\inputminted[breaklines]{c++}{数论/莫比乌斯反演.cpp}

\subsection{数的位数公式}
\inputminted[breaklines]{c++}{数论/数的位数公式.cpp} 

\subsection{辛普森积分}
\inputminted[breaklines]{c++}{数论/辛普森积分.cpp} 

\subsection{矩阵快速幂} 
\inputminted[breaklines]{c++}{数论/矩阵快速幂.cpp} 

\subsection{欧拉函数} 
\inputminted[breaklines]{c++}{数论/欧拉函数.cpp} 

\subsection{欧拉定理} 
\subsubsection{ap互质}
\begin{equation}
a ^ b \% p = (a \% p) ^ {b \% \Phi(p)}  \% p
\end{equation}

\subsubsection{扩展欧拉定理 ap不互质}
\begin{equation}
a ^ b \% p = (a \% p ) ^ {\Phi(p) + b \% \Phi(p)} \% p (b >= \Phi(p))
\end{equation}
\begin{equation}
a ^ b \% p = (a \% p ) ^ b \% p (b < \Phi(p))
\end{equation}

\subsection{常用公式} 

自然数幂次和公式\\
\begin{Large}
\begin{displaymath}

\sum_{i = 1}^{n}i = C_{n + 1}^{2} = \frac{n(n + 1)}{2}\\

\sum_{i = 1}^{n}i^2 = C_{n + 1} ^ {2} + 2C_{n + 1} ^ {3} = \frac{n(n + 1)(2n + 1)}{6}\\

\sum_{i = 1}^{n}i^3 = C_{n + 1} ^ {2} + 6C_{n + 1} ^ {3} + 6C_{n + 1} ^ {4} = \frac{n ^ 2(n + 1) ^ 2}{4}\\

\sum_{i = 1}^{n}i^4 = C_{n + 1}^{2} + 14C_{n + 1}^{3} + 36C_{n + 1}^{4} + 24C_{n + 1}^{5} = \frac{n(n + 1)(6n ^ 3 + 9n ^ 2 + n - 1)}{30}\\

\sum_{i = 1}^{n}i^5 = C_{n + 1} ^ {2} + 30C_{n + 1}^{3} + 150C_{n + 1}^{4} + 240C_{n + 1}^{5} + 120C_{n + 1}^{6} = \frac{n^2(n + 1)(2n^3 + 4n^2+n-1)}{12}\\

\sum_{i = 1}^{n}i^p = \sum_{k = 1} ^ {p}\left[\sum_{j = 0} ^ {k - 1}(-1) ^ jC_{k} ^{j}(k - j) ^ {p + 1}\right]C_{n + 1} ^ {k + 1}\\

\end{displaymath}


斯特林公式\\
\begin{displaymath}
n!\approx \sqrt{2\pi n}\left(\frac{n}{e}\right) ^ {n}
\end{displaymath}

皮克定理\\
2S = 2a + b - 2
\\S:多边形面积,a:多边形内部点数,b:多边形边上点数
\begin{displaymath}

\end{displaymath}

\begin{displaymath}

\end{displaymath}

\begin{displaymath}

\end{displaymath}

\begin{displaymath}

\end{displaymath}

\begin{displaymath}

\end{displaymath}

\begin{displaymath}

\end{displaymath}
\begin{displaymath}

\end{displaymath}
\end{Large}

\section{计算几何}

\subsection{计算几何} 
\inputminted[breaklines]{c++}{计算几何/计算几何.cpp}

\subsection{极角排序} 
\inputminted[breaklines]{c++}{计算几何/极角排序.cpp}

\subsection{凸包} 
\inputminted[breaklines]{c++}{计算几何/凸包.cpp} 

\section{其他} 
\subsection{STL}
\subsubsection{multiset}
\inputminted[breaklines]{c++}{其他/multiset.cpp}

\subsubsection{bitset}
\inputminted[breaklines]{c++}{其他/bitset.cpp}

\subsection{整数三分} 
\inputminted[breaklines]{c++}{其他/整数三分.cpp} 

\subsection{表达式树} 
\inputminted[breaklines]{c++}{其他/表达式树.cpp} 

\subsection{切比雪夫距离} 
\inputminted[breaklines]{c++}{其他/切比雪夫距离.cpp} 

\subsection{离散化} 
\inputminted[breaklines]{c++}{其他/离散化.cpp}

\subsection{区间离散化} 
\inputminted[breaklines]{c++}{其他/区间离散化.cpp}

\subsection{模拟退火} 
\inputminted[breaklines]{c++}{其他/模拟退火.cpp}

\subsection{java大数}
\inputminted[breaklines]{c++}{其他/java大数.cpp}

\subsection{注意事项}
\input{其他/注意事项.txt}

\end{document}
