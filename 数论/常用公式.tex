
自然数幂次和公式\\
\begin{Large}
\begin{displaymath}

\sum_{i = 1}^{n}i = C_{n + 1}^{2} = \frac{n(n + 1)}{2}\\

\sum_{i = 1}^{n}i^2 = C_{n + 1} ^ {2} + 2C_{n + 1} ^ {3} = \frac{n(n + 1)(2n + 1)}{6}\\

\sum_{i = 1}^{n}i^3 = C_{n + 1} ^ {2} + 6C_{n + 1} ^ {3} + 6C_{n + 1} ^ {4} = \frac{n ^ 2(n + 1) ^ 2}{4}\\

\sum_{i = 1}^{n}i^4 = C_{n + 1}^{2} + 14C_{n + 1}^{3} + 36C_{n + 1}^{4} + 24C_{n + 1}^{5} = \frac{n(n + 1)(6n ^ 3 + 9n ^ 2 + n - 1)}{30}\\

\sum_{i = 1}^{n}i^5 = C_{n + 1} ^ {2} + 30C_{n + 1}^{3} + 150C_{n + 1}^{4} + 240C_{n + 1}^{5} + 120C_{n + 1}^{6} = \frac{n^2(n + 1)(2n^3 + 4n^2+n-1)}{12}\\

\sum_{i = 1}^{n}i^p = \sum_{k = 1} ^ {p}\left[\sum_{j = 0} ^ {k - 1}(-1) ^ jC_{k} ^{j}(k - j) ^ {p + 1}\right]C_{n + 1} ^ {k + 1}\\

\end{displaymath}


斯特林公式\\
\begin{displaymath}
n!\approx \sqrt{2\pi n}\left(\frac{n}{e}\right) ^ {n}
\end{displaymath}

皮克定理\\
2S = 2a + b - 2
\\S:多边形面积,a:多边形内部点数,b:多边形边上点数
\begin{displaymath}

\end{displaymath}

\begin{displaymath}

\end{displaymath}

\begin{displaymath}

\end{displaymath}

\begin{displaymath}

\end{displaymath}

\begin{displaymath}

\end{displaymath}

\begin{displaymath}

\end{displaymath}
\begin{displaymath}

\end{displaymath}
\end{Large}